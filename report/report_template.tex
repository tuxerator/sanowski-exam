\documentclass[twocolumn]{article}
\usepackage{graphics}
\usepackage{graphicx}
\usepackage[utf8]{inputenc}
\usepackage{hyperref}
\usepackage{natbib}
\usepackage{amsmath}
\usepackage{algorithm2e}

% \usepackage{themeKonstanz}
% \usepackage{themeKonstanzXelatexAddOn}

\newcommand{\aetitle}{Comparison of an approximation, heuristic and integer-linear program for the max-cut problem on unweighted, undirected graphs} % Title of the report
\newcommand{\studentOne}{Jakob Sanowski} % Name 1
\newcommand{\studentTwo}{} % Name 2


\begin{document}

\twocolumn[{\begin{small}
\begin{minipage}{0.5 \linewidth}
  Algorithm Engineering\\
  WS 21/22
\end{minipage}
\begin{minipage}{0.5\linewidth}
  \begin{flushright}
    \studentOne\\
    \studentTwo
  \end{flushright}
\end{minipage}
\end{small}}
{\begin{center}
\begin{sffamily}\Large\bfseries \aetitle \end{sffamily}
\end{center}}
\vskip 3em]

\begin{abstract}
    The abstract gives a short summary of the project. Begin by stating the motivation of the research at hand, describe the problem and shortly describe what methods you used to solve this problem. Finally, name the most important findings and provide a brief conclusion of your work.
\end{abstract}


\section{Introduction}
% The Introduction is meant to lead the reader into the task at hand. State the motivation and purpose of your project, and name the achieved goals.

\section{Preliminaries}
% The preliminaries provide the reader with necessary background information. In this section the basic algorithms and the ideas behind them are explained.
The maximum-cut of a graph is the cut which size is at least the size of any other possible cut.
A cut is a partition of all vertices into two complementary sets $S$ and $T$,
which size is defined by the number of edges between $S$ and $T$.

For finding an exact solution we can formulate an integer-linear program \(ILP\).
Because the max-cut problem is NP-Hard there do not exist any polynominal-time algorithms for it.
There are .. numerous approximation algorithms and heuristics for getting an approximate solution.
In the following I will briefly explain the ILP-formulation, 0.5-approximation algorithm and heuristic
I used.

\subsection{Integer-Linear Program}
An Integer-Linear program is a maximazation (or minimazation) of a objective function subject to a number of constraints with some or all of the variables being integers.
The objective function and the constraints have to be linear.
The maximum-cut problem can be formulated as follows:

Given a graph $G = (V, E)$, let $\{x_v\}_{v \in V}, \{z_e\}_{e \in E} \in \{0, 1\}$ where $x_v$ encodes the partition $v$ belongs to
and $x_e$ encodes whether $e$ is part of the cut or not.
Then the ILP can be formulated as follows:

\begin{align}
  \text{maximise} \; & \sum_{uv \in E} z_{uv} \\
  \text{subject to:} \; & z_{uv} \leq x_u + x_v, \label{cst1} \\
  & z_{uv} \leq 2 - (x_u + x_v) \label{cst2}
\end{align}

This discribes our problem because $z_{uv}$ can only be $1$ if $uv$ is part of the cut. 
When $u$ and $v$ are in the same partition, $x_u + x_v = 0$ (or $2$ respectively).
Because of (\ref{cst1}) and (\ref{cst2}) $z_{uv}$ then only can be $0$.
Only when $u$ and $v$ are in different partitions $x_u + x_v = 1$ and $z_uv$ can become
$1$ and thus be part of the cut.
This formulation has $|V| + |E|$ variables (for each vertex and edge) and $2 * |E|$ constraints.

\subsection{0.5-Approximation Algorithm}

The 0.5-approximation algorithm is a greedy which iteratively inserts each vertex in the partition
which improves the cut the most. For each vertex $v$ we count the the number of neighbors in
$S$ and $T$ and then insert $v$ in the partition wich contains less neighbors of $v$ in order
to increase the cut size as much as possible. Algorithm~\ref{alg:approx} shows the pseudo-code
of this algorithm.

This algorithm finds a 0.5-approximation because if we let $C_{v_i}$ be the cut with all vertex added
before $v_i$ and $E_{v_i} = \{(v_i, u) | u \in \{v_1, \dots, v_{i -1}\}\}$ the edges between $v_i$
and all vertices added before $v_i$ then at least $\frac{1}{2}|E_{v_i}|$ get added to $C_{v_i}$.

Summing over all $v \in V$ we get:
\begin{align*}
  |C| & \geq \frac{1}{2} \sum_{v} |E_v| \\
  |C| & \geq \frac{1}{2} |E|
\end{align*}




\begin{algorithm}
  \caption{0.5-Approximation Algorithm for Maximum-Cut}
  \label{alg:approx}
  \KwIn{Graph $G = (V, E)$}
  \KwOut{Cut $C$}
  \Begin{
    \ForEach{$v \in V$}{
      \If{$|neigh(v) \in S| \leq |neigh(v) \notin S|$}{
        $S = S \cup \{v\}$
      }
    }
    \ForEach{$e \in E$}{
      \If{$e$ has only one vertex in $S$}{
        $C = C \cup \{e\}$ \\
      }
    }
    \KwRet{C}
  }
\end{algorithm}

 
\subsection{Heuristic}

For the heuristic we can simply do a random flip if we add a vertex to $S$ or not.
This also results in a 0.5-approximation. That is, for each $v \in V$:

\[
  P(v \in S) = P(v \notin S) = \frac{1}{2}
\]

Thus the probability for any edge $(v, u)$ to be in the cut is the probability of $u$ and $v$
beign assigned to different partitions.

\begin{align*}
  P(e \in C) & = P(v \in S \wedge u \notin S) + P(v \notin S \wedge u \in S) \\
  & = \frac{1}{4} + \frac{1}{4} \\
  & = \frac{1}{2}
\end{align*}

The overall expected size of the cut is:

\begin{align*}
  |C| & = \sum_{e \in E} P(e \in C) \\
  & = \frac{1}{2} * |E|
\end{align*}

Which is a 0.5-approximation on expectation.
Algorithm~\ref{alg:heuristic} show the pseudo-code for this heuristic.

\begin{algorithm}
  \caption{0.5-Heuristic for Maximum-Cut}
  \label{alg:heuristic}
  \KwIn{Graph $G = (V, E)$}
  \KwOut{Cut $C$}
  \Begin{
    \ForEach{$v \in V$}{
      randomly add $v$ to $S$ with probability $\frac{1}{2}$ \\
    }
    \ForEach{$e \in E$}{
      \If{$e$ has only one vertex in $S$}{
        $C = C \cup \{e\}$ \\
      }
    }
    \KwRet{C}
  }
\end{algorithm}


\section{Algorithm \& Implementation}
% This section provides information about the actually used algorithms and their respective implementations. It should roughly cover the following three topics:
In order to compare the performance of these three algorithms I have to come up with implementations for these algorithms.
In this chapter I will first talk about possible modifications to the base algorithms which will hopefully increase the performance of them.
Secondly, I will provide the implementation details for the three algorithms.

\subsection{Algorithm Engineering}
For the ILP-formulation I could not find any improvements given it already is very concise.

The approximation algorithm on the other side can be improved by instead of searching for each neighbor of the currently considerd $v$ in $S$
in order to figure out if it is contained in $S$, we can generate a lookup table over all vertex and save whether it got added to $S$ or not.
This not only reduces the practical running time but also the theoretical because instead of having to search in $S$ to find out if a vertex is
in $S$ or not we can simply do a lookup at the position for the vertex in our lookup-table which is in $\mathcal{O}(1)$ instead of $\mathcal{O}(|S|)$.
Furthermore instead of calculating the edges contained in the cut after creating the partitions we can do this during the creation of the partitions.
For each vertex $v$ we just add all edges to neighbors which are not in the same partition as $v$ to the cut.

The theoretical running time for the heuristic on the other hand cannot be reduced. But it is possible to parallise this algorithm very easily.
We can simply divide the vertices into $p$ equaly sized, disjunct partitions, where $p$ is the number of availlable processors.
Each processor then gets assigned one of the partitions and adds each $v$ in the partition to $S$ randomly.

\subsection{Implementation Details}

All algorithms were implemented in Rust(1.67.1).

As a data structure for the graph I used adjacency lists. This datastructure consists of an array over all vertices. 
For each vertex the set of outgoing edges is saved as an array of the indices of it's neighbors reachable via outgoing edges.
I made this decicion because this data structure allows iterating over the neighbors for
a vertex $v$ in $\Theta(deg(v))$ which is highly beneficial for the approximation algorithm.

The ILP was formulated with the libary \href{https://crates.io/crates/good\_lp}{good\_lp}. For the solver \href{d}{solver} was used.
To formulate the problem the libary provides a data structure to which the variables and constraints get added and passes this
formulation to the solver.

For the approximation algorithms no external libraries were used. 
In its basic version I just go over all vertices and iterate through all neighbors of each vertex $v$ to count how many are already contained in $S$.
If less neighbors are in $S$ than there aren't, $v$ gets added to $S$. $S$ is implemented as a growable array holding the indices of all $v$ contained in $S$.
To find all cut edges I iterate over all edges and test if only one of their vertex is in $S$.
The engineered version uses a fixed size array holding booleans for $S$. A vertex is contained in $S$ if the entry at it's index is true.
It also uses a growable array to collect all edges included in the cut. In each round I now count how many neighbors are in $S$ (and aren't respectively) by checking their entry in
the lookup array and add the vertex to $S$ if less neighbors were in $S$ than weren't. Then all edges to neighbors not in $S$ (or in $S$ respectively) get added to the cut.
One important thing is to ignore all neighbors with a higher index than the current vertex or edges that aren't in the final cut might be added.

Lastly, for the heuristic the libary \href{https://crates.io/crates/rand}{rand} was used for random-number-generation.
The for the basic implementation each vertex just randomly gets added to $S$ with probability $\frac{1}{2}$.
In order to parallelise this heuristic all vertices get divided into evenly sized arrays and passed to a seperate thread.
Each thread then applies the basic heuristic to his slice. For generating random numbers a simple pseudo-random number generator is used
in order to increase performance because for this application it is not necessary to use true random numbers.

\begin{itemize}
	\item \textbf{Advanced Algorithm:}\\
		Give and explain the advanced algorithms that you used, and compare them to the basic algorithms.
	\item \textbf{Implementation:}\\
		Explain how you implemented these algorithms and state what external libraries you used.
	\item \textbf{Algorithm Engineering Concepts:}\\
		State the algorithm engineering concepts that you used and explain why they were helpful (if applicable).
\end{itemize}

\section{Experimental Evaluation}
% In this section, the experimental setup is described and the results are presented.
Experiments were conducted on a 4-core (8 hyper-threads) Intel i7-6700K CPU with a clock speed
of 4.00GHz an 16GiB of RAM.

The approximation algorithm and the heuristic were evaluated on 30 graphs of increasing size. 
Table \reg{tbl:graph1} shows the number of nodes and edges for each graph.

The ILP was tested on three relativly small graphs because bigger graphs would take way to long to get a result.
Table \reg{tbl:graph2} shows the size of these graphs.

\subsection{Data and Hardware}%
\label{sub:Data and Hardware}
Here, the input data and the applied parameters are described. Further, information about the underlying hardware such as main memory and CPU is provided.

\subsection{Results}%
\label{sub:Results}
In this part, the results are presented. This includes comparison of running time and memory usage.\\
To visualize the results we recommend one of the following graphing tools:
\begin{itemize}
	\item \textbf{mathplotlib:}\\ 
		\href{https://matplotlib.org/}{matplotlib} is a library for python which allows the direct visualization of the produced results.
	\item \textbf{R:}\\
		R is a programming language for statistical computations, which provides different graphing possibilities. One of the better known is \href{https://ggplot2.tidyverse.org/}{ggplot2}.
	\item \textbf{gnuplot:}\\
		\href{http://www.gnuplot.info/}{gnuplot} is a command-line plotting program. It can also be used to graph data directly from different programming languages, such as Java and C++.
\end{itemize}

\section{Discussion and Conclusion}
In this section, the results are discussed and interpreted. Finally, the work is summarized shortly.

\section{References}
The references list the external resources used in the work at hand. \LaTeX$ $  offers special ways to list those resources. In this template the references are stored in the 'refs.bib' file and can be referenced with the '\textbackslash$ $cite\{REF\}' command, where REF is a label defined in the .bib file. This example shows how such a reference looks like: \cite{exa}.

\bibliographystyle{abbrvnat}
\bibliography{refs}

\end{document}
